% Options for packages loaded elsewhere
\PassOptionsToPackage{unicode}{hyperref}
\PassOptionsToPackage{hyphens}{url}
\PassOptionsToPackage{dvipsnames,svgnames,x11names}{xcolor}
%
\documentclass[
  12pt,
  letterpaper,
  DIV=11,
  numbers=noendperiod]{scrartcl}

\usepackage{amsmath,amssymb}
\usepackage{setspace}
\usepackage{iftex}
\ifPDFTeX
  \usepackage[T1]{fontenc}
  \usepackage[utf8]{inputenc}
  \usepackage{textcomp} % provide euro and other symbols
\else % if luatex or xetex
  \usepackage{unicode-math}
  \defaultfontfeatures{Scale=MatchLowercase}
  \defaultfontfeatures[\rmfamily]{Ligatures=TeX,Scale=1}
\fi
\usepackage{lmodern}
\ifPDFTeX\else  
    % xetex/luatex font selection
\fi
% Use upquote if available, for straight quotes in verbatim environments
\IfFileExists{upquote.sty}{\usepackage{upquote}}{}
\IfFileExists{microtype.sty}{% use microtype if available
  \usepackage[]{microtype}
  \UseMicrotypeSet[protrusion]{basicmath} % disable protrusion for tt fonts
}{}
\makeatletter
\@ifundefined{KOMAClassName}{% if non-KOMA class
  \IfFileExists{parskip.sty}{%
    \usepackage{parskip}
  }{% else
    \setlength{\parindent}{0pt}
    \setlength{\parskip}{6pt plus 2pt minus 1pt}}
}{% if KOMA class
  \KOMAoptions{parskip=half}}
\makeatother
\usepackage{xcolor}
\usepackage[top=1.5cm,bottom=3cm,hmargin=2.5cm]{geometry}
\setlength{\emergencystretch}{3em} % prevent overfull lines
\setcounter{secnumdepth}{5}
% Make \paragraph and \subparagraph free-standing
\makeatletter
\ifx\paragraph\undefined\else
  \let\oldparagraph\paragraph
  \renewcommand{\paragraph}{
    \@ifstar
      \xxxParagraphStar
      \xxxParagraphNoStar
  }
  \newcommand{\xxxParagraphStar}[1]{\oldparagraph*{#1}\mbox{}}
  \newcommand{\xxxParagraphNoStar}[1]{\oldparagraph{#1}\mbox{}}
\fi
\ifx\subparagraph\undefined\else
  \let\oldsubparagraph\subparagraph
  \renewcommand{\subparagraph}{
    \@ifstar
      \xxxSubParagraphStar
      \xxxSubParagraphNoStar
  }
  \newcommand{\xxxSubParagraphStar}[1]{\oldsubparagraph*{#1}\mbox{}}
  \newcommand{\xxxSubParagraphNoStar}[1]{\oldsubparagraph{#1}\mbox{}}
\fi
\makeatother


\providecommand{\tightlist}{%
  \setlength{\itemsep}{0pt}\setlength{\parskip}{0pt}}\usepackage{longtable,booktabs,array}
\usepackage{calc} % for calculating minipage widths
% Correct order of tables after \paragraph or \subparagraph
\usepackage{etoolbox}
\makeatletter
\patchcmd\longtable{\par}{\if@noskipsec\mbox{}\fi\par}{}{}
\makeatother
% Allow footnotes in longtable head/foot
\IfFileExists{footnotehyper.sty}{\usepackage{footnotehyper}}{\usepackage{footnote}}
\makesavenoteenv{longtable}
\usepackage{graphicx}
\makeatletter
\newsavebox\pandoc@box
\newcommand*\pandocbounded[1]{% scales image to fit in text height/width
  \sbox\pandoc@box{#1}%
  \Gscale@div\@tempa{\textheight}{\dimexpr\ht\pandoc@box+\dp\pandoc@box\relax}%
  \Gscale@div\@tempb{\linewidth}{\wd\pandoc@box}%
  \ifdim\@tempb\p@<\@tempa\p@\let\@tempa\@tempb\fi% select the smaller of both
  \ifdim\@tempa\p@<\p@\scalebox{\@tempa}{\usebox\pandoc@box}%
  \else\usebox{\pandoc@box}%
  \fi%
}
% Set default figure placement to htbp
\def\fps@figure{htbp}
\makeatother
% definitions for citeproc citations
\NewDocumentCommand\citeproctext{}{}
\NewDocumentCommand\citeproc{mm}{%
  \begingroup\def\citeproctext{#2}\cite{#1}\endgroup}
\makeatletter
 % allow citations to break across lines
 \let\@cite@ofmt\@firstofone
 % avoid brackets around text for \cite:
 \def\@biblabel#1{}
 \def\@cite#1#2{{#1\if@tempswa , #2\fi}}
\makeatother
\newlength{\cslhangindent}
\setlength{\cslhangindent}{1.5em}
\newlength{\csllabelwidth}
\setlength{\csllabelwidth}{3em}
\newenvironment{CSLReferences}[2] % #1 hanging-indent, #2 entry-spacing
 {\begin{list}{}{%
  \setlength{\itemindent}{0pt}
  \setlength{\leftmargin}{0pt}
  \setlength{\parsep}{0pt}
  % turn on hanging indent if param 1 is 1
  \ifodd #1
   \setlength{\leftmargin}{\cslhangindent}
   \setlength{\itemindent}{-1\cslhangindent}
  \fi
  % set entry spacing
  \setlength{\itemsep}{#2\baselineskip}}}
 {\end{list}}
\usepackage{calc}
\newcommand{\CSLBlock}[1]{\hfill\break\parbox[t]{\linewidth}{\strut\ignorespaces#1\strut}}
\newcommand{\CSLLeftMargin}[1]{\parbox[t]{\csllabelwidth}{\strut#1\strut}}
\newcommand{\CSLRightInline}[1]{\parbox[t]{\linewidth - \csllabelwidth}{\strut#1\strut}}
\newcommand{\CSLIndent}[1]{\hspace{\cslhangindent}#1}

\usepackage{fvextra}
\DefineVerbatimEnvironment{Highlighting}{Verbatim}{
    commandchars=\\\{\},
    breaklines, breaknonspaceingroup, breakanywhere
}
\KOMAoption{captions}{tableheading}
\makeatletter
\@ifpackageloaded{caption}{}{\usepackage{caption}}
\AtBeginDocument{%
\ifdefined\contentsname
  \renewcommand*\contentsname{Table of contents}
\else
  \newcommand\contentsname{Table of contents}
\fi
\ifdefined\listfigurename
  \renewcommand*\listfigurename{List of Figures}
\else
  \newcommand\listfigurename{List of Figures}
\fi
\ifdefined\listtablename
  \renewcommand*\listtablename{List of Tables}
\else
  \newcommand\listtablename{List of Tables}
\fi
\ifdefined\figurename
  \renewcommand*\figurename{Figure}
\else
  \newcommand\figurename{Figure}
\fi
\ifdefined\tablename
  \renewcommand*\tablename{Table}
\else
  \newcommand\tablename{Table}
\fi
}
\@ifpackageloaded{float}{}{\usepackage{float}}
\floatstyle{ruled}
\@ifundefined{c@chapter}{\newfloat{codelisting}{h}{lop}}{\newfloat{codelisting}{h}{lop}[chapter]}
\floatname{codelisting}{Listing}
\newcommand*\listoflistings{\listof{codelisting}{List of Listings}}
\makeatother
\makeatletter
\makeatother
\makeatletter
\@ifpackageloaded{caption}{}{\usepackage{caption}}
\@ifpackageloaded{subcaption}{}{\usepackage{subcaption}}
\makeatother

\usepackage{bookmark}

\IfFileExists{xurl.sty}{\usepackage{xurl}}{} % add URL line breaks if available
\urlstyle{same} % disable monospaced font for URLs
\hypersetup{
  pdftitle={Detecting Digital Imposters: An Expert Analysis of Bot Activity on Reddit},
  pdfauthor={Matteo Mazzarelli},
  colorlinks=true,
  linkcolor={blue},
  filecolor={Maroon},
  citecolor={Blue},
  urlcolor={Blue},
  pdfcreator={LaTeX via pandoc}}


\title{\textbf{Detecting Digital Imposters: An Expert Analysis of Bot
Activity on Reddit}}
\usepackage{etoolbox}
\makeatletter
\providecommand{\subtitle}[1]{% add subtitle to \maketitle
  \apptocmd{\@title}{\par {\large #1 \par}}{}{}
}
\makeatother
\subtitle{Computational Social Science WS2024/25}
\author{Matteo Mazzarelli}
\date{March 24, 2025}

\begin{document}
\maketitle
\begin{abstract}
In the digital age, automated accounts or bots on social media platforms
like Reddit pose a significant threat to online discourse. This paper
investigates the efficacy of basic heuristic methods for detecting bot
influence on Reddit discussions. Employing the Reddit API, we collected
data from five high-traffic subreddits and applied a heuristic-based bot
detection method utilizing meta-metrics such as account age, karma,
posting frequency, content repetitiveness, and em-dash presence.
Exploratory analysis using machine learning classifiers provided
preliminary validation of the heuristic approach in identifying bot-like
accounts based on these meta-metrics. However, keyword frequency and
sentiment analysis, aided by Large Language Models, revealed no
statistically significant content-based differences between accounts
flagged as potential bots and non-flagged accounts, suggesting bots are
increasingly capable of mimicking human language at a surface level.
While heuristics effectively flagged accounts exhibiting bot-like
behavior, they proved insufficient for content-based bot identification
without further nuanced analysis. This study highlights the limitations
of relying solely on meta-metrics and basic content analysis for bot
detection, underscoring the necessity for future research to incorporate
human-validated content labeling and advanced machine learning
techniques capable of discerning subtle linguistic cues in bot-generated
text. The development of content-aware bot detection methods is crucial
for maintaining the integrity of online discussions on platforms like
Reddit.
\end{abstract}

\renewcommand*\contentsname{Table of contents}
{
\hypersetup{linkcolor=}
\setcounter{tocdepth}{3}
\tableofcontents
}

\setstretch{1}
\section{\texorpdfstring{\textbf{1.
Introduction}}{1. Introduction}}\label{introduction}

The proliferation of automated accounts, commonly known as bots, on
social media platforms has become a pervasive issue with the potential
to significantly manipulate online discourse. These bots can be deployed
to influence public opinion, disseminate misinformation, and engage in
various forms of spam and scams, thereby undermining the integrity of
online communities 1. Reddit, with its unique community-based structure
centered around topic-specific subreddits, presents both distinct
challenges and opportunities in the realm of bot detection 1. Notably,
the early history of Reddit even involved the deliberate use of fake
profiles to simulate platform popularity, indicating a long-standing
awareness of the power of artificial engagement 5. The increasing
sophistication of these automated entities, fueled by advancements in
artificial intelligence, makes their accurate identification an ongoing
and complex endeavor 1. This report aims to provide a comprehensive
analysis of the current landscape of Reddit bot detection, encompassing
the methods employed, the datasets utilized, the characteristic features
of bots, the machine learning algorithms applied, the available
open-source tools, the insights shared within the research community,
the inherent limitations and challenges, and the techniques for
collecting and labeling data for this critical task.\\
The early acknowledgment by Reddit's co-founder of using simulated users
to inflate the platform's initial appeal underscores the enduring nature
of artificial activity on the site 5. This historical context is
essential for understanding why bot detection remains a critical
concern. The deliberate introduction of bots in the past to cultivate a
perception of vibrancy might have inadvertently laid the groundwork for
more insidious forms of automated manipulation that persist today.
Furthermore, Reddit's distinct community-driven structure, where
discussions are organized within specialized subreddits, creates a
fragmented environment that can be strategically exploited by bots to
target specific user groups or topics 1. This necessitates detection
strategies that are sensitive to the nuances of individual communities.
The continuous progress in artificial intelligence, particularly in
natural language generation, has dramatically enhanced the ability of
social bots to mimic human-like communication, thereby demanding
constant innovation in detection methodologies to effectively
distinguish between authentic users and sophisticated automated accounts
1.

\section{\texorpdfstring{\textbf{2. Academic Landscape of Reddit Bot
Detection}}{2. Academic Landscape of Reddit Bot Detection}}\label{academic-landscape-of-reddit-bot-detection}

The academic research dedicated to bot detection has evolved
significantly over time, progressing from initial feature-based
approaches to more intricate methodologies encompassing temporal
analysis and the examination of user interaction networks 9.
Feature-based methods concentrate on identifying specific attributes or
patterns associated with bot accounts, such as unusual username
structures or repetitive content. Temporal methods analyze the timing
and frequency of user activities, looking for patterns that suggest
automation, like rapid bursts of posts or comments. Graph-based methods
delve into the social connections and interaction patterns between
users, aiming to identify coordinated or anomalous network behaviors
indicative of bot activity. Several research papers have specifically
addressed the challenge of bot detection on Reddit.\\
The study ``Bot Detection in Reddit Political Discussion'' provides a
detailed characterization of suspicious behavior on the platform,
leveraging interaction-driven engagement data and proposing machine
learning solutions grounded in the analysis of social graphs and user
metadata 1. This research also explores the subtle ways in which bots
can exert influence, even indirectly, through platform recommendation
systems. Furthermore, an ensemble method designed for multi-platform bot
detection, encompassing Twitter, Reddit, and Instagram, has been
developed, emphasizing the use of minimal feature engineering and
optimized classifiers 9. This approach prioritizes the ability to handle
incomplete data and aims for generalizability across different social
media environments. Other research efforts have focused on utilizing the
structural characteristics of user interaction networks on Reddit to
identify bots 16. Moreover, numerous studies have investigated the role
and detection of bots in specific events, such as elections and protests
9. The development of publicly available tools like Botometer represents
another key area of research, providing platforms for evaluating the
likelihood of an account being automated 12. Additionally, research has
explored the detection of fake news dissemination by bots on Reddit,
recognizing the platform's role in information sharing 1. Finally,
multimodal approaches that integrate diverse information sources,
including text, images, and user statistics, are being increasingly
investigated to enhance detection accuracy 1. The field has also seen a
shift in focus from predominantly supervised learning techniques, which
rely on pre-labeled data of bots and humans, towards unsupervised
learning methods that aim to discover patterns and anomalies without the
need for explicit labels 8.\\
The increasing emphasis in academic research on detecting bots that
operate across multiple platforms reflects a growing understanding that
bot networks often extend beyond the confines of a single social media
site 9. This cross-platform perspective suggests that a more effective
approach to detection might involve analyzing user behavior and
connections across various online environments. The specific focus of
research on interaction patterns and social graphs within Reddit's
subreddits highlights the platform's unique characteristics and the
potential for developing tailored detection methods 1. Given Reddit's
distinct structure, detection techniques designed specifically for this
platform might offer greater accuracy compared to generic bot detection
tools. Furthermore, the significant body of research dedicated to
identifying bots in political discussions and elections underscores the
profound societal implications of automated manipulation and the
critical importance of research in these sensitive areas 9. This
research focus reflects the urgency of mitigating the potential negative
impacts of bots on democratic processes and public opinion formation.

\section{\texorpdfstring{\textbf{3. Publicly Accessible Datasets for
Training and
Evaluation}}{3. Publicly Accessible Datasets for Training and Evaluation}}\label{publicly-accessible-datasets-for-training-and-evaluation}

A variety of publicly accessible datasets are available for researchers
seeking to train and evaluate algorithms designed to detect bot activity
on Reddit 9. The Cresci datasets, including cresci-rtbust-2019 and
cresci-stock-2018, often contain user names, screen names, descriptions,
and in some cases, posts and associated metadata 9. These datasets
typically include a mix of bot and human accounts, with varying
proportions. Botometer-feedback-2019 and Botwiki-2019 provide data
specifically labeled to distinguish between bot and human accounts 9.
Midterms-2018 and Political-bots-2019 focus on bot activity related to
political discourse, offering different levels of data availability
across various user features 9. The Reddit Comments Dataset, hosted on
Clickhouse, is a massive repository containing billions of Reddit
comments spanning from 2005 to 2023 20. This dataset includes
information such as the subreddit, author, comment body, and timestamps,
although it does not inherently label bots. However, it serves as a
valuable resource for extracting features and potentially labeling data
for bot detection research. The Reddit Conversations Dataset, available
on Kaggle, comprises conversational exchanges from specific subreddits
like r/CasualConversation 22. While primarily intended for chatbot
training, this data could be further processed and labeled for bot
detection purposes. Several other similar conversation datasets also
exist on Kaggle 23. The Pushshift Reddit Dataset represents a
comprehensive archive of Reddit comments and submissions, accessible
through an API 16. This platform allows researchers to collect
customized datasets based on specific criteria, making it an invaluable
resource for large-scale data acquisition. The Reddit Bot Detector
Dataset, associated with an open-source bot detection project on GitHub,
potentially contains labeled bot and human accounts, offering a direct
resource for training and evaluation 31. Additionally, some datasets
focus on specific user behaviors, such as mouse tracking data collected
in CAPTCHA systems, which can indirectly contribute to bot detection
strategies by identifying non-human interaction patterns 32. It is worth
noting that historically, a significant portion of academic bot
detection research has concentrated on data from Twitter 33.\\
While a considerable number of datasets are available, the scarcity of
large-scale, high-quality datasets \emph{specifically labeled} for bot
detection on Reddit presents a potential challenge compared to platforms
like Twitter 9. Researchers may often need to dedicate substantial
effort to the process of collecting and labeling Reddit-specific data.
In this context, the Pushshift API emerges as a particularly valuable
asset due to its extensive historical data and the flexibility it offers
in data collection 16. The ability to access data dating back to
Reddit's inception, coupled with the API's search and aggregation
functionalities, makes it an indispensable tool for investigating
long-term trends in bot activity or focusing on specific events.
Furthermore, the diversity in the types of available data, ranging from
individual comments and conversational threads to user profiles and
network structures, suggests that different research questions and
detection methodologies may necessitate the use of different datasets or
combinations thereof 9, etc.{]}. This variety allows for a multifaceted
approach to studying bot behavior, whether the focus is on analyzing
user characteristics, content patterns, or the dynamics of online
interactions.\\
The following table summarizes some of the key publicly available Reddit
bot detection datasets:

\begin{longtable}[]{@{}
  >{\raggedright\arraybackslash}p{(\linewidth - 10\tabcolsep) * \real{0.1667}}
  >{\raggedright\arraybackslash}p{(\linewidth - 10\tabcolsep) * \real{0.1667}}
  >{\raggedright\arraybackslash}p{(\linewidth - 10\tabcolsep) * \real{0.1667}}
  >{\raggedright\arraybackslash}p{(\linewidth - 10\tabcolsep) * \real{0.1667}}
  >{\raggedright\arraybackslash}p{(\linewidth - 10\tabcolsep) * \real{0.1667}}
  >{\raggedright\arraybackslash}p{(\linewidth - 10\tabcolsep) * \real{0.1667}}@{}}
\toprule\noalign{}
\begin{minipage}[b]{\linewidth}\raggedright
Dataset Name
\end{minipage} & \begin{minipage}[b]{\linewidth}\raggedright
Source
\end{minipage} & \begin{minipage}[b]{\linewidth}\raggedright
Description
\end{minipage} & \begin{minipage}[b]{\linewidth}\raggedright
Bot Labels Available
\end{minipage} & \begin{minipage}[b]{\linewidth}\raggedright
Timeframe
\end{minipage} & \begin{minipage}[b]{\linewidth}\raggedright
Key Features
\end{minipage} \\
\midrule\noalign{}
\endhead
\bottomrule\noalign{}
\endlastfoot
botometer-feedback-2019 & Yang et al. & Dataset used for feedback on
Botometer predictions. & Yes & 2019 & User name, screen name,
description, partial posts, partial user metadata. \\
botwiki-2019 & Yang et al. & Dataset of accounts identified as bots on
Botwiki. & Yes (100\% bots) & 2019 & User name, screen name,
description, partial posts, partial user metadata. \\
cresci-rtbust-2019 & Mazza et al. & Dataset related to real-time
bursting events and potential bot activity. & Yes & 2019 & User name,
screen name, partial posts, partial user metadata. \\
cresci-stock-2018 & Cresci et al. & Dataset focused on fake accounts
involved in stock market manipulation. & Yes & 2018 & User name, screen
name, partial posts, partial user metadata. \\
midterms-2018 & Yang et al. & Dataset related to bot activity during the
2018 US midterm elections. & Yes & 2018 & User name, screen name,
partial posts, partial user metadata. \\
political-bots-2019 & Yang et al. & Dataset of political bots. & Yes
(100\% bots) & 2019 & User name, screen name, description, posts, user
metadata. \\
Reddit Comments Dataset & Clickhouse & Massive dataset of publicly
available Reddit comments. & No & 2005-2023 & Subreddit, subreddit ID,
subreddit type, author, body, timestamps, link ID, score, awards,
etc. \\
Reddit Conversations Dataset & Kaggle & Conversations from
r/CasualConversation. & No & 2016-2019 & Length-3 conversations. \\
Pushshift Reddit Dataset & Pushshift & Comprehensive archive of Reddit
comments and submissions. & No & 2005-Present & All publicly available
Reddit data (comments, submissions, metadata). \\
Reddit-Bot-Detector Dataset & GitHub & Dataset associated with the
Reddit-Bot-Detector project. & Potentially & Varies & Comment history
and potentially labels. \\
\end{longtable}

This table provides a structured overview of the datasets discussed,
enabling researchers to quickly identify resources relevant to their
specific needs based on the availability of bot labels, the timeframe of
the data, and the key features included in each dataset.

\section{\texorpdfstring{\textbf{4. Identifying the Digital Imposters:
Features and Characteristics of Reddit
Bots}}{4. Identifying the Digital Imposters: Features and Characteristics of Reddit Bots}}\label{identifying-the-digital-imposters-features-and-characteristics-of-reddit-bots}

Identifying bot activity on Reddit often involves analyzing a
combination of features and characteristics associated with user
accounts and their interactions. These indicators can be broadly
categorized to provide a structured approach to detection.\\
Username patterns can offer initial clues. Bots frequently employ
usernames that consist of random strings of letters and numbers or
follow a pattern of two human-sounding names, often feminine, appended
with a series of letters, such as ``sss'' 2. Another common pattern is
the use of default Reddit-generated usernames, which typically comprise
two words separated by a hyphen followed by four numbers 2. Variations
of these default names, lacking the hyphen or the numerical suffix, are
also observed 3. Additionally, some bots utilize generic or nonsensical
usernames 2.\\
Account activity and age are also significant indicators. Bot accounts
are often relatively new, frequently less than a year old 3. However,
some bot operators might intentionally create older accounts to
circumvent age-based restrictions in certain subreddits 2. A sudden
surge in activity from an account that has been dormant for an extended
period can also be suspicious, potentially indicating a stolen or aged
account being repurposed for bot activity 38. High posting frequency or
the occurrence of multiple comments within very short timeframes, even
across different subreddits within the same minute, strongly suggests
automated behavior 2.\\
The content of comments and posts often reveals bot characteristics.
Bots frequently post very short, generic, and contextually ambiguous
phrases that seem out of place in the conversation 2. Copy-pasting
comments, either from within the same thread or from previous instances
of the same content, is a common tactic 2. These copied comments might
sometimes be slightly altered or have simple phrases like ``Yeah'' added
3. Incomplete or out-of-context comments can also be indicative of
poorly implemented copy-paste bots 2. Some bots generate comments by
blending existing phrases, sometimes resulting in grammatically awkward
sentences 40. Issues with punctuation, such as unpaired quotation marks
or parentheses, can also be a sign of automated content generation 40.
With the rise of sophisticated AI, some bots generate ``wholesome''
comments with perfect grammar and punctuation, sometimes including
emojis 2. Bots often concentrate their activity in specific subreddits,
such as r/AskReddit, to farm karma 2. Spamming the same comment multiple
times is another common tactic 2. Reposting old, popular content,
sometimes with minor modifications to evade detection, is also
frequently observed 2. An inability to correctly process basic symbols
in reposted titles can be a strong indicator of a bot 2. Many bots are
designed to promote products or scams, often with coordinated efforts
involving multiple bots 2. These bots might link to specific websites,
such as Gearlaunch for fraudulent merchandise, or use screenshots to
avoid automatic link detection 2.\\
Interaction patterns can also be revealing. Bots often reply to the top
or second most upvoted comments in a thread and tend to make direct
replies rather than engaging in extended comment chains 3. Blocking
users who identify and report them as bots is another common behavior 2.
Coordinated activity among groups of bots, such as one bot posting
content and others immediately commenting, is also frequently observed
3.\\
Finally, profile information can provide additional clues. Bots often
lack profile pictures or use the default randomized Snoo avatar 2. Their
profile descriptions might contain links to pornographic or scam
websites 2. While some bots might have low post or comment karma, others
are specifically designed to accumulate karma to appear more legitimate
2. It is crucial to recognize that accurately identifying bots typically
requires considering a combination of these factors rather than relying
on any single characteristic in isolation 2.\\
The shift in bot username patterns from easily recognizable random
strings to more human-like formats demonstrates an adaptive response by
bot creators to basic detection techniques 2. This evolution
necessitates that detection algorithms move beyond simple username
analysis. The strategic ``aging'' of accounts, where bots initially
engage in normal interactions before transitioning to malicious
activities, highlights a sophisticated tactic designed to build trust
and bypass account age filters 39. This emphasizes the need for
detection methods to consider the entire history of an account's
behavior. The increasing reliance of bots on AI-generated content,
characterized by specific linguistic traits, presents a significant
challenge, as these bots can produce grammatically sound and
contextually relevant text 2. This necessitates the development of
detection techniques capable of identifying subtle patterns inherent in
AI-generated language.

\section{\texorpdfstring{\textbf{5. Leveraging Machine Learning for Bot
Detection}}{5. Leveraging Machine Learning for Bot Detection}}\label{leveraging-machine-learning-for-bot-detection}

Machine learning algorithms play a crucial role in the automated
detection of bot activity on Reddit 9. Various approaches are employed,
broadly categorized as classification models and anomaly detection
techniques.\\
Classification models, based on supervised learning, aim to categorize
users as either bots or humans by learning from labeled data 9.
Tree-based classifiers, such as Decision Trees and Random Forests, have
been utilized due to their computational efficiency and interpretability
9. Ensemble methods, which combine the predictions of multiple
classifiers, can enhance both accuracy and robustness in bot detection
9. For instance, an ensemble method employing tree-based classifiers
achieved an overall accuracy of 75.47\% across multiple social media
platforms, including Reddit 13. Deep learning models, particularly Long
Short-Term Memory (LSTM) networks, are also being explored for their
ability to capture complex temporal dependencies in user activity
patterns 8.\\
Anomaly detection techniques, which fall under unsupervised learning,
focus on identifying users whose behavior deviates significantly from
the established norm 2. Histogram-Based Outlier Scoring (HBOS) is one
such algorithm used for scalable anomaly detection in large datasets 49.
A key advantage of anomaly detection is its capacity to identify novel
bot behaviors that have not been encountered during training 49.\\
The effectiveness of these machine learning models heavily relies on the
careful selection and engineering of relevant features 1. These features
can encompass a wide range of attributes, including user metadata (e.g.,
account creation date, follower/following counts), posting frequency,
characteristics of the content (e.g., sentiment, use of URLs or
hashtags), and network-based features derived from user interactions.
Some specialized tools, like Bot-Sleuth-Bot, utilize a ``Suspicion
Quotient'' to provide a probabilistic assessment of an account being a
bot, based on a combination of analyzed features 51.\\
It is important to note that the reported accuracy of bot detection
models can vary considerably. While some studies claim high accuracy
rates, such as 93\% on Twitter data using BERT, the performance on
Reddit data may differ 1. One study focused on Reddit bot detection
achieved an accuracy of 91.7\% using a decision tree classifier 44.
Another research effort, utilizing the CrediRAG model, reported an 11\%
increase in the F1-score for detecting misinformation, which is often
associated with bot activity 1.\\
The increasing adoption of ensemble methods in bot detection suggests
that combining the strengths of various algorithms and feature sets can
lead to more robust and accurate detection systems 9. The success of
multi-platform ensemble approaches indicates that a diverse collection
of classifiers, each specializing in different aspects of user data, can
be more effective than relying on a single model. This is likely because
bots exhibit a wide array of behaviors, and no single algorithm may be
capable of capturing all of them comprehensively. While supervised
learning has been a prominent technique, the growing sophistication of
bots and the challenges in obtaining extensive labeled data are driving
interest in unsupervised anomaly detection methods 2. Anomaly detection
offers the advantage of identifying new and previously unseen bot
behaviors that might not be present in the training data for supervised
models, which is particularly relevant in the context of the ongoing
``arms race'' between bot developers and detection researchers. The
performance of bot detection models can vary significantly depending on
the specific platform and the characteristics of the targeted bots 1.
High accuracy achieved on one platform, such as Twitter, does not
guarantee similar results on Reddit, suggesting that bot detection
models may need to be specifically trained and evaluated for the unique
environment of Reddit to achieve optimal performance.

\section{\texorpdfstring{\textbf{6. The Toolkit: Open-Source Resources
for Reddit Bot
Analysis}}{6. The Toolkit: Open-Source Resources for Reddit Bot Analysis}}\label{the-toolkit-open-source-resources-for-reddit-bot-analysis}

A range of open-source tools, libraries, and platforms are available to
facilitate the analysis and detection of bot activity on Reddit. Python,
with its rich ecosystem of data science libraries, is a particularly
popular choice for this task. Libraries such as Scikit-learn provide
comprehensive machine learning algorithms for building and evaluating
bot detection models 44. Pandas is essential for data manipulation and
analysis 44, while TextBlob offers natural language processing
capabilities, including sentiment analysis, which can be used as a
feature in bot detection 44. For numerical and scientific computations,
NumPy and SciPy are widely used 52. Data visualization libraries like
Plotly, Matplotlib, and Seaborn are valuable for exploring patterns and
trends in bot behavior 52.\\
Several open-source projects are specifically focused on Reddit bot
detection. Reddit-Bot-Detector on GitHub is a Python bot that identifies
bots based on their comment history, using metrics such as the cosine
similarity of their posts, posting frequency, median reply time, and
reply patterns within comment chains 31. Bot-Sleuth-Bot is a Reddit bot
that analyzes user accounts and provides a probability score indicating
the likelihood of the account being a bot, based on publicly available
profile data 38. Another project, creme332/reddit-spam-bot-detector,
offers a basic algorithm for detecting spam bots on Reddit using
heuristics such as account age and posting frequency 42.\\
For collecting data from Reddit, especially when building custom
datasets, web scraping tools can be useful. Scrapfly is a service
designed to help users avoid bot detection while scraping websites,
offering features like proxy rotation and anti-scraping protection 20.
Puppeteer, a Node library for browser automation, can be used to scrape
dynamic content that might be challenging to access through APIs 20.\\
Data analysis platforms like Clickhouse, which hosts a massive Reddit
comments dataset, provide the infrastructure for efficiently querying
and analyzing large volumes of Reddit data 21. Additionally, third-party
Reddit applications, such as Infinity for Reddit, offer features like
precise comment timestamps, which can aid in identifying coordinated bot
activity 2. Finally, open-source security solutions like open-appsec,
which provide machine learning-based threat protection for web
applications and APIs, could potentially be adapted for detecting bot
activity on Reddit 57.\\
The prominent role of Python libraries in this toolkit underscores the
language's significance in data analysis and machine learning within the
field of bot detection 44. The repeated use of libraries like
Scikit-learn, Pandas, and TextBlob highlights Python's robust ecosystem
for developing bot detection solutions. The existence of dedicated
open-source projects for Reddit bot detection, such as
Reddit-Bot-Detector and creme332/reddit-spam-bot-detector 31,
demonstrates a collaborative community effort in creating and sharing
tools for this purpose. The inclusion of web scraping tools like
Scrapfly and Puppeteer 20 emphasizes the necessity of data collection,
especially for unsupervised methods or creating custom labeled datasets,
given the inherent challenges in scraping Reddit due to the platform's
own bot detection mechanisms 20.

\section{\texorpdfstring{\textbf{7. Community Insights: Discussions and
Shared
Resources}}{7. Community Insights: Discussions and Shared Resources}}\label{community-insights-discussions-and-shared-resources}

Online forums and communities, particularly on Reddit itself, serve as
valuable platforms for researchers and developers to share insights and
resources related to Reddit bot detection. Subreddits like r/botwatch
provide a dedicated space for discussing detection strategies and
sharing observations about bot behavior 33. Communities such as
r/LearnUselessTalents and r/Humanornot often host discussions where
users share anecdotal evidence and observed patterns to identify bots 2.
The subreddit r/webscraping features discussions on techniques for
circumventing bot detection when scraping data, which indirectly sheds
light on the methods platforms use to identify bots 20.
r/MachineLearning occasionally hosts discussions on the application of
machine learning to bot detection in social networks, including Reddit
36. r/redditdev offers insights into how Reddit's platform and API can
be utilized for bot detection purposes 41. Moderator communities, such
as r/ModSupport, discuss tools and strategies for managing bot activity,
including the use of anti-repost bots 43. Individual Reddit users
frequently share their personal methods for identifying bots based on
observable characteristics like username patterns, comment content, and
overall behavior 2. These discussions also delve into the motivations
behind bot activity, such as farming karma, spreading misinformation,
and executing scams 2. Furthermore, the use of third-party applications
like Infinity for Reddit is mentioned as a helpful tool for gaining more
detailed information, such as precise comment timestamps, which can aid
in detecting coordinated bot activity 2.\\
Community discussions on Reddit provide a valuable reservoir of
practical, often experience-based, knowledge about identifying bots,
which can serve as a complement to formal academic research. These
real-world observations can yield valuable features and heuristics for
enhancing bot detection strategies 2. The challenges discussed within
web scraping communities regarding evading bot detection offer valuable
insights into the signals that platforms like Reddit might be using to
identify automated activity 20. Understanding these anti-scraping
techniques can inform the development of more effective bot detection
algorithms. The sharing of specific bot behaviors within the Reddit
community, such as the promotion of scam websites or the use of
particular posting patterns, enables a collective effort in identifying
and potentially reporting malicious bots 2. This collaborative aspect of
bot detection can be a powerful tool in mitigating the impact of
automated manipulation on the platform.

\section{\texorpdfstring{\textbf{8. Navigating the Obstacles:
Limitations and Challenges in Accurate Bot
Identification}}{8. Navigating the Obstacles: Limitations and Challenges in Accurate Bot Identification}}\label{navigating-the-obstacles-limitations-and-challenges-in-accurate-bot-identification}

Accurately identifying bots on Reddit presents several inherent
limitations and ongoing challenges 9. One of the primary difficulties is
the constantly evolving sophistication of bots. Bot creators continually
adapt their techniques to evade detection mechanisms, resulting in a
continuous ``arms race'' between bot developers and those seeking to
identify them 1. Increasingly, sophisticated bots can effectively mimic
human behavior, especially with advancements in artificial intelligence,
making it challenging to distinguish them from genuine users 1.\\
Overly aggressive bot detection methods can also lead to the problem of
false positives, where legitimate human users are incorrectly classified
as bots 2. This can cause frustration among users and potentially impact
their engagement with the platform. Furthermore, what might be
considered bot-like behavior in one context may be perfectly normal for
a highly active human user in another, highlighting the importance of
context-dependent analysis. Researchers may also face limitations in
accessing comprehensive data from Reddit due to privacy concerns or
restrictions imposed by the platform 8. Analyzing the vast amounts of
data generated on Reddit for bot detection purposes can be
computationally intensive, requiring significant resources 9. Ethical
considerations are also paramount, as bot detection methods must be
carefully designed to avoid unfairly targeting specific user groups 72.
Bots designed to exhibit low activity levels over extended periods,
aiming to blend in with regular users, can be particularly challenging
to detect 2. Finally, the ability of bots to understand and respond to
context with human-like language, including nuances like sarcasm or
humor, remains a significant hurdle in accurate identification 2.\\
The continuous adaptation of bot techniques necessitates a dynamic and
adaptive approach to bot detection. Static rules or models are likely to
become ineffective over time as bot creators discover new ways to
circumvent them 1. The fundamental challenge of balancing detection
accuracy with the risk of false positives requires careful consideration
when designing bot detection systems 2. Setting overly stringent
detection thresholds might increase the number of identified bots but
could also lead to the incorrect flagging of legitimate users. The
increasing proficiency of bots in generating human-like text using
advanced AI models suggests that traditional content-based detection
methods may become less reliable 2. Future research may need to place
greater emphasis on analyzing behavioral patterns and network
interactions to effectively identify these more sophisticated automated
accounts.

\section{\texorpdfstring{\textbf{9. Building the Foundation: Techniques
for Data Collection and
Labeling}}{9. Building the Foundation: Techniques for Data Collection and Labeling}}\label{building-the-foundation-techniques-for-data-collection-and-labeling}

The process of collecting and labeling Reddit data is fundamental to the
development and evaluation of effective bot detection algorithms 9. Data
collection typically involves utilizing the Reddit API, often through
libraries like PRAW, to gather posts, comments, and user information.
The Pushshift API is particularly valuable for accessing historical
Reddit data, providing a comprehensive archive for research purposes 16.
In some cases, web scraping techniques might be employed to collect data
not readily available through APIs, although this approach requires
caution due to Reddit's bot detection measures 20.\\
Data labeling, the process of categorizing data points as either bot or
human, can be achieved through various methods. Manual labeling involves
human examination of user profiles and activity based on known bot
characteristics to assign labels 2. While providing accurate labels,
this method can be time-consuming and susceptible to human error,
especially when dealing with sophisticated bots 7. Another approach
involves utilizing existing lists of known bot accounts, often compiled
by research projects or bot detection initiatives 44. However, these
lists may not be exhaustive or entirely up-to-date. Heuristic-based
labeling applies predefined rules based on observed bot characteristics
to automatically categorize data 1. This method can be efficient but
might miss more nuanced bot behaviors. Crowdsourcing platforms like
Amazon Mechanical Turk can be used to engage human annotators for
labeling, although this requires careful quality control and can incur
costs 73. AI-assisted labeling leverages pre-trained AI models or
semi-supervised learning techniques to automate a significant portion of
the labeling process, with subsequent manual review and correction 71.
Tools like Originality.ai, designed to detect AI-generated content, can
also aid in identifying potential bot activity, as many modern bots rely
on AI for content creation 39. Finally, data derived from user reports
of suspected bots can be used as a form of labeling, although this data
might be noisy and require verification 2. A significant challenge in
this process is obtaining truly accurate ground truth labels,
particularly with the increasing sophistication of bots that can closely
mimic human behavior 7.\\
The most effective strategy for creating large and reasonably accurate
datasets for Reddit bot detection likely involves a combination of
automated data collection through APIs and a blend of manual and
AI-assisted labeling techniques. While automated collection using APIs
is essential for handling the scale of Reddit data, AI-assisted labeling
can significantly accelerate the labeling process. However, human
oversight remains crucial for ensuring the accuracy of labels,
especially when dealing with advanced bots. Given the evolving nature of
bots and their ability to imitate human users, obtaining perfect ground
truth labels is inherently difficult. Therefore, evaluation metrics for
bot detection models should account for this uncertainty, and research
should explore methods for handling noisy labels. The emergence of tools
specifically designed to detect AI-generated content offers a promising
new direction for bot detection, as many contemporary bots rely on such
techniques to produce realistic text. The ability to identify
AI-generated content can serve as a strong indicator of automated
account activity.

\section{\texorpdfstring{\textbf{10. Conclusion and Future Research
Directions}}{10. Conclusion and Future Research Directions}}\label{conclusion-and-future-research-directions}

In conclusion, the detection of bot activity on Reddit is a multifaceted
challenge that demands a comprehensive understanding of bot
characteristics, the application of advanced analytical techniques, and
ongoing collaboration within the research community. This report has
explored the academic landscape of Reddit bot detection, highlighting
key research efforts and the evolution of detection methodologies. It
has also identified publicly available datasets that serve as crucial
resources for training and evaluating detection algorithms. A detailed
examination of the features and characteristics commonly associated with
Reddit bots provides a foundation for developing effective
identification strategies. Furthermore, the application of various
machine learning algorithms, including classification and anomaly
detection techniques, demonstrates the power of data-driven approaches
in this domain. The availability of open-source tools and the insights
shared within online communities underscore the collaborative nature of
this research area. Despite these advancements, significant limitations
and challenges remain, particularly with the increasing sophistication
of bots and the difficulty in obtaining accurate ground truth labels.
Various techniques for data collection and labeling have been discussed,
highlighting the need for robust and scalable methodologies.\\
Future research in Reddit bot detection should increasingly focus on
developing more robust and adaptive algorithms capable of keeping pace
with the evolving tactics of bot creators, potentially leveraging
advanced deep learning techniques and graph neural networks. A greater
emphasis should be placed on detecting coordinated bot activity and
identifying entire bot networks rather than focusing solely on
individual accounts. Improving the methods for collecting and labeling
high-quality datasets remains critical, with exploration into active
learning and semi-supervised learning approaches. Investigating the
effectiveness of different mitigation strategies for addressing
identified bots is also an important area for future work. Furthermore,
research should delve deeper into understanding the specific impact of
bots on various communities and types of discourse on Reddit. The
ethical implications of bot detection and mitigation strategies warrant
careful consideration. The development of real-time bot detection
systems that can proactively identify and flag malicious activity is
another crucial direction. Finally, establishing benchmarks and
standardized evaluation protocols for Reddit bot detection models would
facilitate comparison and accelerate progress in the field. The ongoing
research and development in this area are of paramount importance for
maintaining the integrity and trustworthiness of online social platforms
like Reddit in the face of increasingly sophisticated automated
manipulation.

\subsubsection{\texorpdfstring{\textbf{Bibliografia}}{Bibliografia}}\label{bibliografia}

\begin{enumerate}
\def\labelenumi{\arabic{enumi}.}
\tightlist
\item
  Bot Detection in Reddit Political Discussion \textbar{} Request PDF -
  ResearchGate, accesso eseguito il giorno marzo 21, 2025,
  \url{https://www.researchgate.net/publication/332340547_Bot_Detection_in_Reddit_Political_Discussion}\\
\item
  How to Identify Bots on Reddit : r/LearnUselessTalents, accesso
  eseguito il giorno marzo 21, 2025,
  \url{https://www.reddit.com/r/LearnUselessTalents/comments/15tzjkb/how_to_identify_bots_on_reddit/}\\
\item
  Bots. How to identify them, and why do they exist on Reddit? :
  u/tyrannosnorlax, accesso eseguito il giorno marzo 21, 2025,
  \url{https://www.reddit.com/user/tyrannosnorlax/comments/t0h466/bots_how_to_identify_them_and_why_do_they_exist/}\\
\item
  On Reddit, what is a ``bot account'', and why do people create them?
  What's the motivation?, accesso eseguito il giorno marzo 21, 2025,
  \url{https://www.reddit.com/r/NoStupidQuestions/comments/1bz33wa/on_reddit_what_is_a_bot_account_and_why_do_people/}\\
\item
  Does Reddit have a bot problem? Absolutely. - Lunio, accesso eseguito
  il giorno marzo 21, 2025,
  \url{https://www.lunio.ai/blog/reddit-bots}\\
\item
  Social Media Bots - Definition, Purpose, Preventive Measures -
  Indusface, accesso eseguito il giorno marzo 21, 2025,
  \url{https://www.indusface.com/learning/social-media-bots/}\\
\item
  Experimental Evaluation: Can Humans Recognise Social Media Bots? -
  MDPI, accesso eseguito il giorno marzo 21, 2025,
  \url{https://www.mdpi.com/2504-2289/8/3/24}\\
\item
  Evaluation of social bot detection models - ResearchGate, accesso
  eseguito il giorno marzo 21, 2025,
  \url{https://www.researchgate.net/publication/361038547_Evaluation_of_social_bot_detection_models}\\
\item
  Assembling a Multi-Platform Ensemble Social Bot Detector with
  Applications to US 2020 Elections - arXiv, accesso eseguito il giorno
  marzo 21, 2025, \url{https://arxiv.org/html/2401.14607v1}\\
\item
  {[}2401.14607{]} Assembling a Multi-Platform Ensemble Social Bot
  Detector with Applications to US 2020 Elections - arXiv, accesso
  eseguito il giorno marzo 21, 2025,
  \url{https://arxiv.org/abs/2401.14607}\\
\item
  Assembling a Multi-Platform Ensemble Social Bot Detector with
  Applications to US 2020 Elections - arXiv, accesso eseguito il giorno
  marzo 21, 2025, \url{https://arxiv.org/pdf/2401.14607}\\
\item
  (PDF) Assembling a multi-platform ensemble social bot detector with
  applications to US 2020 elections - ResearchGate, accesso eseguito il
  giorno marzo 21, 2025,
  \url{https://www.researchgate.net/publication/378464325_Assembling_a_multi-platform_ensemble_social_bot_detector_with_applications_to_US_2020_elections}\\
\item
  Assembling a Multi-Platform Ensemble Social Bot Detector with
  Applications to US 2020 Elections - arXiv, accesso eseguito il giorno
  marzo 21, 2025, \url{https://arxiv.org/html/2401.14607v2}\\
\item
  BotBuster: Multi-platform Bot Detection Using A Mixture of Experts -
  Semantic Scholar, accesso eseguito il giorno marzo 21, 2025,
  \url{https://www.semanticscholar.org/paper/BotBuster\%3A-Multi-platform-Bot-Detection-Using-A-of-Ng-Carley/d4bfa40f79c6b0f519c17755431732e0d76f0df6}\\
\item
  Tiny-BotBuster: Identifying Automated Political Coordination in
  Digital Campaigns \textbar{} Request PDF - ResearchGate, accesso
  eseguito il giorno marzo 21, 2025,
  \url{https://www.researchgate.net/publication/382723202_Tiny-BotBuster_Identifying_Automated_Political_Coordination_in_Digital_Campaigns}\\
\item
  DamascenoRafael/identify-bots-reddit-comment-network: Characterization
  and classification of bots using only structural characteristics of
  the network. Python development of network construction, component
  analysis and Neural Network for classification. - GitHub, accesso
  eseguito il giorno marzo 21, 2025,
  \url{https://github.com/DamascenoRafael/identify-bots-reddit-comment-network}\\
\item
  (PDF) Political Social Media Bot Detection: Unveiling Cutting-edge
  Feature Selection and Engineering Strategies in Machine Learning Model
  Development - ResearchGate, accesso eseguito il giorno marzo 21, 2025,
  \url{https://www.researchgate.net/publication/380948046_Political_Social_Media_Bot_Detection_Unveiling_Cutting-edge_Feature_Selection_and_Engineering_Strategies_in_Machine_Learning_Model_Development}\\
\item
  {[}2312.17423{]} Social Bots: Detection and Challenges - arXiv,
  accesso eseguito il giorno marzo 21, 2025,
  \url{https://arxiv.org/abs/2312.17423}\\
\item
  Detecting bots in social-networks using node and structural embeddings
  - PMC, accesso eseguito il giorno marzo 21, 2025,
  \url{https://pmc.ncbi.nlm.nih.gov/articles/PMC10356665/}\\
\item
  Avoiding Bot Detection : r/webscraping - Reddit, accesso eseguito il
  giorno marzo 21, 2025,
  \url{https://www.reddit.com/r/webscraping/comments/shhr1s/avoiding_bot_detection/}\\
\item
  Reddit comments dataset \textbar{} ClickHouse Docs, accesso eseguito
  il giorno marzo 21, 2025,
  \url{https://clickhouse.com/docs/getting-started/example-datasets/reddit-comments}\\
\item
  Reddit Conversations - Kaggle, accesso eseguito il giorno marzo 21,
  2025,
  \url{https://www.kaggle.com/datasets/jerryqu/reddit-conversations}\\
\item
  Reddit Conversation Dataset - Kaggle, accesso eseguito il giorno marzo
  21, 2025,
  \url{https://www.kaggle.com/datasets/psyflow/reddit-conversation-dataset}\\
\item
  1 million Reddit comments from 40 subreddits - Kaggle, accesso
  eseguito il giorno marzo 21, 2025,
  \url{https://www.kaggle.com/datasets/smagnan/1-million-reddit-comments-from-40-subreddits}\\
\item
  Twitter and Reddit Sentimental analysis Dataset - Kaggle, accesso
  eseguito il giorno marzo 21, 2025,
  \url{https://www.kaggle.com/datasets/cosmos98/twitter-and-reddit-sentimental-analysis-dataset}\\
\item
  View of The Pushshift Reddit Dataset - AAAI Publications, accesso
  eseguito il giorno marzo 21, 2025,
  \url{https://ojs.aaai.org/index.php/ICWSM/article/view/7347/7201}\\
\item
  {[}2001.08435{]} The Pushshift Reddit Dataset - arXiv, accesso
  eseguito il giorno marzo 21, 2025,
  \url{https://arxiv.org/abs/2001.08435}\\
\item
  Pushshift API - GitHub, accesso eseguito il giorno marzo 21, 2025,
  \url{https://github.com/pushshift/api}\\
\item
  Pushshift Reddit Dataset - Papers With Code, accesso eseguito il
  giorno marzo 21, 2025,
  \url{https://paperswithcode.com/dataset/pushshift-reddit}\\
\item
  Pushshift.io, accesso eseguito il giorno marzo 21, 2025,
  \url{https://pushshift.io/}\\
\item
  MatthewTourond/Reddit-Bot-Detector: A Python bot that detects Reddit
  bots - GitHub, accesso eseguito il giorno marzo 21, 2025,
  \url{https://github.com/MatthewTourond/Reddit-Bot-Detector}\\
\item
  Mouse Tracking for Bot Detection in CAPTCHA Systems : r/datasets -
  Reddit, accesso eseguito il giorno marzo 21, 2025,
  \url{https://www.reddit.com/r/datasets/comments/1f0ncua/mouse_tracking_for_bot_detection_in_captcha/}\\
\item
  r/botwatch - Reddit, accesso eseguito il giorno marzo 21, 2025,
  \url{https://www.reddit.com/r/botwatch/}\\
\item
  Publicly available dataset recommendations : r/epidemiology - Reddit,
  accesso eseguito il giorno marzo 21, 2025,
  \url{https://www.reddit.com/r/epidemiology/comments/m7gxfb/publicly_available_dataset_recommendations/}\\
\item
  What are some good publicly available real-time data sources? :
  r/datasets - Reddit, accesso eseguito il giorno marzo 21, 2025,
  \url{https://www.reddit.com/r/datasets/comments/13vuof8/what_are_some_good_publicly_available_realtime/}\\
\item
  {[}D{]} Bot detection in a social network : r/MachineLearning -
  Reddit, accesso eseguito il giorno marzo 21, 2025,
  \url{https://www.reddit.com/r/MachineLearning/comments/12qs4fy/d_bot_detection_in_a_social_network/}\\
\item
  The Reddit Dataset Dataset - Kaggle, accesso eseguito il giorno marzo
  21, 2025,
  \url{https://www.kaggle.com/datasets/pavellexyr/the-reddit-dataset-dataset}\\
\item
  Bot Problem - How to Identify Bot Accounts (99\% Accuracy) :
  r/7daystodie - Reddit, accesso eseguito il giorno marzo 21, 2025,
  \url{https://www.reddit.com/r/7daystodie/comments/1au1g1s/bot_problem_how_to_identify_bot_accounts_99/}\\
\item
  Most people on Reddit might not even be people \textbar{} by Sohail
  Saha - Medium, accesso eseguito il giorno marzo 21, 2025,
  \url{https://captain-woof.medium.com/most-people-on-reddit-might-not-even-be-people-2b207a7f1902}\\
\item
  how do you spot a bot? : r/TheoryOfReddit, accesso eseguito il giorno
  marzo 21, 2025,
  \url{https://www.reddit.com/r/TheoryOfReddit/comments/wuc8qx/how_do_you_spot_a_bot/}\\
\item
  How do I check if a user is a bot? - Reddit, accesso eseguito il
  giorno marzo 21, 2025,
  \url{https://www.reddit.com/r/redditdev/comments/lhi2wn/how_do_i_check_if_a_user_is_a_bot/}\\
\item
  creme332/reddit-spam-bot-detector - GitHub, accesso eseguito il giorno
  marzo 21, 2025,
  \url{https://github.com/creme332/reddit-spam-bot-detector}\\
\item
  Introducing DuplicateDestroyer 2.0 : an improved repost bot with text
  detection - Reddit, accesso eseguito il giorno marzo 21, 2025,
  \url{https://www.reddit.com/r/ModSupport/comments/10bbfa4/introducing_duplicatedestroyer_20_an_improved/}\\
\item
  Identifying trolls and bots on Reddit with machine learning (Part 2) -
  Medium, accesso eseguito il giorno marzo 21, 2025,
  \url{https://medium.com/towards-data-science/identifying-trolls-and-bots-on-reddit-with-machine-learning-709da5970af1}\\
\item
  Oyebamiji-Micheal/Detection-of-Social-Bots-using-Machine-Learning -
  GitHub, accesso eseguito il giorno marzo 21, 2025,
  \url{https://github.com/Oyebamiji-Micheal/Detection-of-Social-Bots-using-Machine-Learning}\\
\item
  How is anomaly detection used in recommendation systems? - Milvus,
  accesso eseguito il giorno marzo 21, 2025,
  \url{https://milvus.io/ai-quick-reference/how-is-anomaly-detection-used-in-recommendation-systems}\\
\item
  how does anomaly detection work : r/f5networks - Reddit, accesso
  eseguito il giorno marzo 21, 2025,
  \url{https://www.reddit.com/r/f5networks/comments/13jw4qg/how_does_anomaly_detection_work/}\\
\item
  Latest anomaly detection techniques for a time series data :
  r/deeplearning - Reddit, accesso eseguito il giorno marzo 21, 2025,
  \url{https://www.reddit.com/r/deeplearning/comments/199jrzo/latest_anomaly_detection_techniques_for_a_time/}\\
\item
  Lessons Learned from Scaling Up Cloudflare's Anomaly Detection
  Platform, accesso eseguito il giorno marzo 21, 2025,
  \url{https://blog.cloudflare.com/lessons-learned-from-scaling-up-cloudflare-anomaly-detection-platform/}\\
\item
  {[}2411.06626{]} Exploring social bots: A feature-based approach to
  improve bot detection in social networks - arXiv, accesso eseguito il
  giorno marzo 21, 2025, \url{https://arxiv.org/abs/2411.06626}\\
\item
  Bot Sleuth Bot - Reddit, accesso eseguito il giorno marzo 21, 2025,
  \url{https://www.reddit.com/user/bot-sleuth-bot/}\\
\item
  Top 10 Python Libraries for Data Analytics \textbar{} Classes Near Me
  Blog - Noble Desktop, accesso eseguito il giorno marzo 21, 2025,
  \url{https://www.nobledesktop.com/classes-near-me/blog/top-python-libraries-for-data-analytics}\\
\item
  What python libraries do you personally recommend for data analyst? :
  r/analytics - Reddit, accesso eseguito il giorno marzo 21, 2025,
  \url{https://www.reddit.com/r/analytics/comments/z143o5/what_python_libraries_do_you_personally_recommend/}\\
\item
  What are the Python packages you consistently use to do data analysis?
  - Reddit, accesso eseguito il giorno marzo 21, 2025,
  \url{https://www.reddit.com/r/Python/comments/16czwre/what_are_the_python_packages_you_consistently_use/}\\
\item
  10 Python Libraries Every Data Analyst Should Know - KDnuggets,
  accesso eseguito il giorno marzo 21, 2025,
  \url{https://www.kdnuggets.com/10-python-libraries-every-data-analyst-should-know}\\
\item
  What are the most important uses of Python for data analytics? :
  r/dataanalysis - Reddit, accesso eseguito il giorno marzo 21, 2025,
  \url{https://www.reddit.com/r/dataanalysis/comments/16yyt8q/what_are_the_most_important_uses_of_python_for/}\\
\item
  Top 10 Bot Detection Tools for 2024, accesso eseguito il giorno marzo
  21, 2025,
  \url{https://www.openappsec.io/post/bot-detection-tools-2024}\\
\item
  How to Scrape Reddit with BrowserQL - Browserless, accesso eseguito il
  giorno marzo 21, 2025,
  \url{https://www.browserless.io/blog/scrape-reddit}\\
\item
  How to get around high-cost scraping of heavily bot detected sites? :
  r/webscraping - Reddit, accesso eseguito il giorno marzo 21, 2025,
  \url{https://www.reddit.com/r/webscraping/comments/1hlzynt/how_to_get_around_highcost_scraping_of_heavily/}\\
\item
  www.reddit.com, accesso eseguito il giorno marzo 21, 2025,
  \href{https://www.reddit.com/r/Humanornot/comments/1db00pi/what_are_your_sure_fire_methods_of_detecting_bots/\#:~:text=Saying\%20random\%20things\%20or\%20phrases,then\%20it's\%20probably\%20a\%20bot.}{https://www.reddit.com/r/Humanornot/comments/1db00pi/what\_are\_your\_sure\_fire\_methods\_of\_detecting\_bots/\#:\textasciitilde:text=Saying\%20random\%20things\%20or\%20phrases,then\%20it's\%20probably\%20a\%20bot.}\\
\item
  Best open source automated bot? : r/screeps - Reddit, accesso eseguito
  il giorno marzo 21, 2025,
  \url{https://www.reddit.com/r/screeps/comments/1abuvog/best_open_source_automated_bot/}\\
\item
  Please explain exactly what is a Bot on Reddit? : r/NoStupidQuestions,
  accesso eseguito il giorno marzo 21, 2025,
  \url{https://www.reddit.com/r/NoStupidQuestions/comments/15h3034/please_explain_exactly_what_is_a_bot_on_reddit/}\\
\item
  Best way to identify bot traffic? : r/node - Reddit, accesso eseguito
  il giorno marzo 21, 2025,
  \url{https://www.reddit.com/r/node/comments/1anao75/best_way_to_identify_bot_traffic/}\\
\item
  how to identify reddit bots ? : r/NewToReddit, accesso eseguito il
  giorno marzo 21, 2025,
  \url{https://www.reddit.com/r/NewToReddit/comments/1e6vall/how_to_identify_reddit_bots/}\\
\item
  What are you bot detection methods? : r/Eve - Reddit, accesso eseguito
  il giorno marzo 21, 2025,
  \url{https://www.reddit.com/r/Eve/comments/186ulk9/what_are_you_bot_detection_methods/}\\
\item
  Deep Research is hands down the best research tool I've used---anyone
  else making the switch? : r/ChatGPTPro - Reddit, accesso eseguito il
  giorno marzo 21, 2025,
  \url{https://www.reddit.com/r/ChatGPTPro/comments/1iis4wy/deep_research_is_hands_down_the_best_research/}\\
\item
  Are You Getting Advice from a Human or Bot? Reddit Shows Spikes in AI
  Content, accesso eseguito il giorno marzo 21, 2025,
  \url{https://originality.ai/blog/reddit-shows-spikes-in-ai-content}\\
\item
  Are you a bot? (AI Model Testing) - flask - Reddit, accesso eseguito
  il giorno marzo 21, 2025,
  \url{https://www.reddit.com/r/flask/comments/1ieo1pj/are_you_a_bot_ai_model_testing/}\\
\item
  Machine learning bot detection software : r/Twitch - Reddit, accesso
  eseguito il giorno marzo 21, 2025,
  \url{https://www.reddit.com/r/Twitch/comments/ds6435/machine_learning_bot_detection_software/}\\
\item
  Suggestions for screening bots out? : r/ProlificAc - Reddit, accesso
  eseguito il giorno marzo 21, 2025,
  \url{https://www.reddit.com/r/ProlificAc/comments/1ewp7dp/suggestions_for_screening_bots_out/}\\
\item
  AI auto labeled/supervised learning image labeling tools question :
  r/computervision - Reddit, accesso eseguito il giorno marzo 21, 2025,
  \url{https://www.reddit.com/r/computervision/comments/17bkpcc/ai_auto_labeledsupervised_learning_image_labeling/}\\
\item
  What is a Bot? Types, Mitigation \& Challenges - SentinelOne, accesso
  eseguito il giorno marzo 21, 2025,
  \url{https://www.sentinelone.com/cybersecurity-101/cybersecurity/what-is-a-bot/}\\
\item
  {[}Discussion{]} What is your go to technique for labelling data? :
  r/MachineLearning - Reddit, accesso eseguito il giorno marzo 21, 2025,
  \url{https://www.reddit.com/r/MachineLearning/comments/powmw5/discussion_what_is_your_go_to_technique_for/}
\end{enumerate}

\phantomsection\label{refs}
\begin{CSLReferences}{1}{0}
\bibitem[\citeproctext]{ref-botdetectionreddit}
1. Hurtado S., Ray P., \& Marculescu R. (2019). Bot detection in reddit
political discussion. \emph{ResearchGate}.
\url{https://www.researchgate.net/publication/332340547_Bot_Detection_in_Reddit_Political_Discussion}

\bibitem[\citeproctext]{ref-redditbotproblem}
2. Dawson A. (2024). Does reddit have a bot problem? absolutely. In
\emph{Lunio}. \url{https://www.lunio.ai/blog/reddit-bots}

\bibitem[\citeproctext]{ref-redditbotwatch}
3. R/botwatch. (2022). In \emph{Reddit}.
\url{https://www.reddit.com/r/botwatch/}

\bibitem[\citeproctext]{ref-multibotdetector}
4. Ng L. H. X., \& Carley K. M. (2022). Assembling a multi-platform
ensemble social bot detector with applications to US 2020 elections.
\emph{arXiv}. \url{https://arxiv.org/html/2401.14607v1}

\bibitem[\citeproctext]{ref-evalsocialbotmodels}
5. Kutlu M., \& Selçuk A. A. (2025). Evaluation of social bot detection
models. \emph{ResearchGate}.
\url{https://www.researchgate.net/publication/361038547_Evaluation_of_social_bot_detection_models}

\bibitem[\citeproctext]{ref-anomalycloudflare}
6. Tang J. (2024). \emph{Lessons learned from scaling up cloudflare's
anomaly detection platform}.
\url{https://blog.cloudflare.com/lessons-learned-from-scaling-up-cloudflare-anomaly-detection-platform/}

\bibitem[\citeproctext]{ref-anomalyf5networks}
7. Tang J. (2024). How does anomaly detection work? : R/f5networks. In
\emph{Reddit}.
\url{https://www.reddit.com/r/f5networks/comments/13jw4qg/how_does_anomaly_detection_work/}

\bibitem[\citeproctext]{ref-pushshiftredditdataset}
8. Baumgartner J., Zannettou S., Keegan B., Squire M., \& Blackburn J.
(2020). The pushshift reddit dataset. \emph{AAAI Publications}.
\url{https://ojs.aaai.org/index.php/ICWSM/article/view/7347/7201}

\bibitem[\citeproctext]{ref-redditcommentsdataset}
9. Tkachenko V. (2021). Reddit comments dataset. In \emph{ClickHouse
Docs}.
\url{https://clickhouse.com/docs/getting-started/example-datasets/reddit-comments}

\bibitem[\citeproctext]{ref-redditbotslearnusetalents}
10. How to identify bots on reddit : R/LearnUselessTalents. (2023). In
\emph{Reddit}.
\url{https://www.reddit.com/r/LearnUselessTalents/comments/15tzjkb/how_to_identify_bots_on_reddit/}

\bibitem[\citeproctext]{ref-botsidentifytyrannosnorlax}
11. u/tyrannosnorlax. (2022). Bots. How to identify them, and why do
they exist on reddit? In \emph{Reddit}.
\url{https://www.reddit.com/user/tyrannosnorlax/comments/t0h466/bots_how_to_identify_them_and_why_do_they_exist/}

\bibitem[\citeproctext]{ref-botproblem7daystodie}
12. Bot problem - how to identify bot accounts (99\% accuracy) :
R/7daystodie. (2025). In \emph{Reddit}.
\url{https://www.reddit.com/r/7daystodie/comments/1au1g1s/bot_problem_how_to_identify_bot_accounts_99/}

\bibitem[\citeproctext]{ref-originalityaiblog}
13. Gillham J., \& Lambert M. (2023). \emph{Are you getting advice from
a human or bot? Reddit shows spikes in AI content}.
\url{https://originality.ai/blog/reddit-shows-spikes-in-ai-content}

\bibitem[\citeproctext]{ref-nightwateremdash}
14. Bush C. (2023). \emph{The em dash and AI: A conjunction - night
water}. \url{https://www.nightwater.email/em-dash-ai/}

\bibitem[\citeproctext]{ref-redditbotdetector}
15. Tourond M. (2019). Reddit-bot-detector: A python bot that detects
reddit bots. In \emph{GitHub}.
\url{https://github.com/MatthewTourond/Reddit-Bot-Detector}

\bibitem[\citeproctext]{ref-redditbotnetwork}
16. Damasceno R. (2019). Identify-bots-reddit-comment-network:
Characterization and classification of bots using only structural
characteristics of the network. In \emph{GitHub}.
\url{https://github.com/DamascenoRafael/identify-bots-reddit-comment-network}

\bibitem[\citeproctext]{ref-mlredditbots}
17. Skowronski J. (2019). Identifying trolls and bots on reddit with
machine learning. In \emph{Medium}.
\url{https://medium.com/towards-data-science/identifying-trolls-and-bots-on-reddit-with-machine-learning-709da5970af1}

\bibitem[\citeproctext]{ref-breiman2001random}
18. Breiman L. (2001). Random forests. \emph{Machine Learning},
\emph{45}(1), 5--32. \url{https://doi.org/10.1023/A:1010933404324}

\bibitem[\citeproctext]{ref-breiman1984classification}
19. Breiman L., Friedman J. H., Olshen R. A., \& Stone C. J. (1984).
\emph{Classification and regression trees}. Wadsworth International
Group.

\bibitem[\citeproctext]{ref-hutto2014vader}
20. Hutto C. J., \& Gilbert E. (2014). \emph{VADER: A parsimonious
rule-based model for sentiment analysis of social media text}.
\url{http://comp.social.gatech.edu/papers/icwsm14.vader.hutto.pdf}

\bibitem[\citeproctext]{ref-salton1988term}
21. Salton G., \& Buckley C. (1988). Term-weighting approaches in
automatic text retrieval. \emph{Information Processing \& Management},
\emph{24}(5), 513--523.
\url{https://doi.org/10.1016/0306-4573(88)90021-0}

\bibitem[\citeproctext]{ref-cortes1995support}
22. Cortes C., \& Vapnik V. (1995). Support-vector networks.
\emph{Machine Learning}, \emph{20}(3), 273--297.
\url{https://doi.org/10.1007/BF00994018}

\bibitem[\citeproctext]{ref-lecun2015deep}
23. LeCun Y., Bengio Y., \& Hinton G. (2015). Deep learning.
\emph{Nature}, \emph{521}(7553), 436--444.
\url{https://doi.org/10.1038/nature14539}

\bibitem[\citeproctext]{ref-kincaid1975flesch}
24. Kincaid J. P., Fishburne Jr R. P., Rogers R. L., \& Chissom B. S.
(1975). \emph{Derivation of new readability formulas (automated
readability index, fog count and flesch reading ease formula) for navy
enlisted personnel}.

\bibitem[\citeproctext]{ref-googleai2023gemini}
25. Google. (2023). \emph{Gemini API}.
\url{https://ai.google.dev/products/gemini}

\bibitem[\citeproctext]{ref-praw2024}
26. Reddit. (2024). \emph{{PRAW}: Python reddit API wrapper}.
\url{https://praw.readthedocs.io/en/stable/}

\bibitem[\citeproctext]{ref-hochreiter1997long}
27. Hochreiter S., \& Schmidhuber J. (1997). Long short-term memory.
\emph{Neural Computation}, \emph{9}(8), 1735--1780.
\url{https://doi.org/10.1162/neco.1997.9.8.1735}

\bibitem[\citeproctext]{ref-dietterich2000ensemble}
28. Dietterich T. G. (2000). \emph{Ensemble methods in machine
learning}. 1--15.

\bibitem[\citeproctext]{ref-chandola2009anomaly}
29. Chandola V., Banerjee A., \& Kumar V. (2009). Anomaly detection: A
survey. \emph{ACM Computing Surveys (CSUR)}, \emph{41}(3), 1--58.
\url{https://doi.org/10.1145/1541880.1541882}

\bibitem[\citeproctext]{ref-goldstein2012histogram}
30. Goldstein M., \& Dengel A. (2012). Histogram-based outlier score
(HBOS): A fast unsupervised anomaly detection algorithm. \emph{KI-2012:
Poster and Demo Track}, 59--63.

\end{CSLReferences}




\end{document}
